\documentclass[10pt,oneside]{article}

\usepackage{amsfonts}
\usepackage{amsmath}
\DeclareMathOperator*{\amax}{arg\,max}
\DeclareMathOperator*{\amin}{arg\,min}
\usepackage{amssymb}
\usepackage{dsfont}
\usepackage{bm}

\usepackage{epsf}
\usepackage{epsfig}
\usepackage{graphicx}
\usepackage{wrapfig} \usepackage{subfig}

\usepackage{enumerate}
\usepackage{listings}

\usepackage{setspace}
\usepackage{geometry}
\usepackage{fancyhdr}
%\usepackage{soul} % cross out text

\usepackage[latin2]{inputenc}
% \usepackage{times} % ez kiszedi a t1enc raszteressgt, de valami ms betu"tpust
% hasznl
\usepackage{lmodern} % ez eltu"nteti a raszteressget s mg jk is a betu"k
% \usepackage[magyar]{babel}
\usepackage{t1enc}

% \usepackage[T1]{fontenc}

\usepackage[usenames]{color}
\usepackage[colorlinks]{hyperref} 
\hypersetup{linkcolor=blue}
% \usepackage{showkeys}

% \onehalfspacing
\usepackage{indentfirst}
% \frenchspacing

\geometry{left=2.5cm,right=2.5cm,top=3.0cm,bottom=2.5cm}

\pagestyle{fancy}
\lhead{mufit2 model}
\chead{ }
\rhead{\thepage}

\lfoot{ }
\cfoot{ }
\rfoot{P\'{e}ter K\'{o}m\'{a}r, \the\year}


\renewcommand{\headrulewidth}{0.4pt}
\renewcommand{\footrulewidth}{0.0pt}

\newcommand{\refeq}[1]{Eq.(\ref{#1})}


% \numberwithin{equation}{section} \numberwithin{figure}{section}
% \numberwithin{table}{section}

\author{Peter Komar}
\title{mufit2 model}
\date{\today}




\begin{document}
\newcommand{\bel}{\begin{equation}}
\newcommand{\eel}{\end{equation}}
\newcommand{\be}{\begin{equation*}}
\newcommand{\ee}{\end{equation*}}

\newcommand{\bal}{\begin{eqnarray}}
\newcommand{\eal}{\end{eqnarray}}
\newcommand{\ba}{\begin{eqnarray*}}
\newcommand{\ea}{\end{eqnarray*}}

\newcommand{\ket}[1]{| #1 \rangle}
\newcommand{\Ket}[1]{\left| #1 \right\rangle}
\newcommand{\bra}[1]{\langle #1 |}
\newcommand{\Bra}[1]{\left\langle #1 \right|}

\newcommand{\no}{\noindent}

\newcommand{\ev}[1]{\langle #1 \rangle}
\newcommand{\Ev}[1]{\left\langle #1 \right\rangle}
\newcommand{\Tr}{\text{Tr}\,}
\newcommand{\T}{^\top}
\newcommand{\+}{^\dagger}
\newcommand{\s}{^\ast}
\newcommand{\PP}{\mathcal{P}}
\newcommand{\eqE}{= \!\!\!\!\!^{{}^{E}}\,}

\renewcommand{\d}[1]{\!d #1 \;}

\newcommand{\bE}{{\mathbf E}}
\newcommand{\bB}{{\mathbf B}}
\newcommand{\bF}{{\mathbf F}}
\newcommand{\bJ}{{\mathbf J}}
\newcommand{\bv}{{\mathbf v}}
\newcommand{\eps}{\varepsilon}
\newcommand{\br}{\mathbf r}
\newcommand{\bk}{\mathbf k}
\newcommand{\hatx}{\hat{\mathbf{x}}}
\newcommand{\haty}{\hat{\mathbf{y}}}
\newcommand{\hatz}{\hat{\mathbf{z}}}

\newcommand\independent{\protect\mathpalette{\protect\independenT}{\perp}}
\def\independenT#1#2{\mathrel{\rlap{$#1#2$}\mkern2mu{#1#2}}}


\renewcommand{\vec}[1]{{\bf #1}}
\newcommand{\mat}[1]{{\bf #1}}

\newcommand{\op}[1]{\mathbf{#1}}
\newcommand{\twovector}[2]{
	\left[
		\begin{array}{c}
		#1 \\
		#2
		\end{array}
	\right]
}
\newcommand{\threevector}[3]{
	\left[
		\begin{array}{c}
		#1 \\
		#2 \\
		#3
		\end{array}
	\right]
}
\newcommand{\fourvector}[4]{
	\left[
		\begin{array}{c}
		#1 \\
		#2 \\
		#3 \\
		#4
		\end{array}
	\right]
}
\newcommand{\fivevector}[5]{
	\left[
		\begin{array}{c}
		#1 \\
		#2 \\
		#3 \\
		#4 \\
		#5
		\end{array}
	\right]
}
\newcommand{\nvector}[2]{
	\left[
		\begin{array}{c}
		#1_1 \\
		#1_2 \\
		\vdots \\
		#1_#2
		\end{array}
	\right]
}
\newcommand{\ncovector}[2]{
	[#1_1\s, #1_2\s, \dots #1_#2\s]
}
\newcommand{\twobytwomatrix}[4]{
	\left[
		\begin{array}{cc}
		#1 & #2\\
		#3 & #4
		\end{array}
	\right]
}
\newcommand{\threebythreematrix}[9]{
	\left[
		\begin{array}{ccc}
		#1 & #2 & #3\\
		#4 & #5 & #6\\
		#7 & #8 & #9
		\end{array}
	\right]
}
\newcommand{\threebythreedeterminant}[9]{
	\left|
		\begin{array}{ccc}
		#1 & #2 & #3\\
		#4 & #5 & #6\\
		#7 & #8 & #9
		\end{array}
	\right|
}
\newcommand{\nbymmatrix}[3]{
	\left[ 
		\begin{array}{cccc}
		#1_{11}  & #1_{12} & \dots  & #1_{1 #2}\\
		#1_{21}  & #1_{22} &        &         \\
		\vdots   &         & \ddots & \vdots  \\
		#1_{#3 1}&         & \dots  & #1_{#3 #2}
		\end{array}
	\right]
}
\newcommand{\nbyndeterminant}[2]{
	\left|
		\begin{array}{cccc}
		#1_{11}  & #1_{12} & \dots  & #1_{1 #2}\\
		#1_{21}  & #1_{22} &        &         \\
		\vdots   &         & \ddots & \vdots  \\
		#1_{#2 1}&         & \dots  & #1_{#2 #2}
		\end{array}
	\right|
}


\newcommand{\argmax}[1]{\underset{#1}{\operatorname{arg}\operatorname{max}}\;}

\thispagestyle{empty}
\maketitle

We introduce a 2-level model, ``hidden Gaussian process'', for smoothening the inferred growth rate curves of optical density time series produced by a turbidostat.

 
%\newpage
\tableofcontents
\newpage

\lstset{
numbers=left, 
numberstyle=\small, 
numbersep=8pt, 
frame = single, 
language=Python, 
framexleftmargin=15pt
}


\section{Data}

\subsection{Raw optical density data}
We start with the time series of the raw optical density (OD) measurements recorded during a single experimental run of a turbidostat. At consecutive (but not necessarily equidistant) time points, OD is recorded. This produces two vectors of real numbers:
\begin{itemize}
	\item the time points, $\{\text{tp}_n\;:\; n = 1,2, \ldots N_\text{total}\}$, and
	\item the OD values, $\{\text{od}_n\;:\; n = 1,2, \ldots N_\text{total}\}$.
\end{itemize}
Under normal operating conditions, the time series (tp, od) has the following features:
\begin{itemize}
	\item a long initial growth from a low OD value to the operating OD regime,
	\item sharp drops of OD, when it reaches a predefined threshold value,
	\item gradual growth of OD between sharps drops, and
	\item intermittent spikes of OD.
\end{itemize}

\subsection{Preprocessing}
Out of the four features of the raw (tp, od) time series, we wish to model only the gradual growth (and maybe the initial growth) phases. For this we filter the time series and partition it into non-overlapping regions by 
\begin{enumerate}
	\item Using heuristic filters to identify sudden changes of od, and remove the associated data points.
	\item Group uninterrupted series of data points into non-overlapping regions.
	\item Take the logarithm of OD.
\end{enumerate}
This produces the cleaned time series $(t, x)$ of $N$ data points:
\ba
	t &=& (t_1, t_2, \ldots t_N), \quad \text{where}\; t_n \in \mathds{R},\quad \text{and}\; t_n < t_{n+1}, \\
	x &=& (x_1, x_2, \ldots x_N), \quad \text{where}\; x_n = \log(\text{od}) \in \mathds{R},
\ea
and a list of $R$ regions, i.e. non-overlapping sets $\mathcal{R}_r$,
\be
	r \in \{1, 2, \ldots R\},\qquad \text{where each }\mathcal{R}_r = \{s(r), s(r) + 1, \ldots e(r) - 1, e(r)\} \subset \{1, 2,\ldots N\}
\ee
is a list of consecutive indexes, where $s(r)$ is the first and $e(r)$ is the last.

\section{Model}

We model all $R$ regions of the cleaned time series $(t, x)$ with a single model that captures the gradual growth within each region as well as two features of between-region change of the growth rate ($\mu$), sudden changes and long-term drifts.

\subsection{Parameters}
We use the following unknown variables to describe the time series. The role of each value will become more clear in the next subsection.
\begin{itemize}
	\item $x_0 = \{x_{r,0}\;:\;r=1,2,\ldots R\} \in \mathds{R}^R$ represents the starting log-OD value at the beginning of each region.
	\item $\mu_1 = \{\mu_{r,1}\;:\;r=1,2,\ldots R\} \in \mathds{R}^R$  represents the growth rate at the beginning of each region.
	\item $\mu_2 = \{\mu_{r,2}\;:\;r=1,2,\ldots R\} \in \mathds{R}^R$  represents the growth rate at the end of each region.
	\item $\sigma_x \in \mathds{R}$ is the strength of measurement noise of log-OD.
	\item $\mu_0$ is $\mu$ at the beginning of the experiment.
	\item $\nu_0$ is the \emph{rate of change} of $\mu$ at the beginning of the experiment
	\item $D$ is the diffusion coefficient of the velocity of the long term drift of $\mu$.
	\item $\tau$ is the time scale of the short-term fluctuations of $\mu$.
	\item $\sigma_y$ is the magnitude of the short-term fluctuations of $\mu$.
\end{itemize}


\subsection{Individual regions}
We assume that for given $x_0, \mu_1, \mu_2$ values, the log-OD values in different regions become independent. This allows us to write the generating distribution of log-OD as 
\be
	P(x\;|\;t, x_0, \mu_1, \mu_2) = \prod_{r=1}^R P(x^{(r)}\;|\;t^{(r)}, x_{r,0}, \mu_{r,1}, \mu_{r,2})\quad ,
\ee
where $x^{(r)}$ and $t^{(r)}$ are the set of log-OD and time points belonging to region $r$.

Furthermore, within each region $r$, we assume that $\mu$ changes deterministically and linearly between $\mu_{r,1}$ (at $t_{s(r)}$) and $\mu_{r,2}$ (at $t_{e(r)}$), i.e.
\be
	\mu(t) = 
	\mu_{r,1} \frac{t_{e(r)} - t}{t_{e(r)} - t_{s(r)}} + 
	\mu_{r,2} \frac{t - t_{s(r)}}{t_{e(r)} - t_{s(r)}}\quad,\quad\text{if}\quad t_{s(r)} \leq t \leq t_{e(r)}\quad.
\ee
This leads to a quadratic time dependence for the ``noiseless'' log-OD,
\ba
	x_r^\text{noiseless}(t) 
	&=& 
	x_{r,0} + \intop_{t_{s(r)}}^{t} \d{t'} \mu(t') = 
	x_{r,0} + f_r(t)\,\mu_{r,1} + g_r(t)\,\mu_{r,2}\quad ,\text{ where }
	\\
	f_r(t) &=& \frac{t_{e(r)}(t - t_{s(r)}) - \frac{1}{2} (t^2 - t_{s(r)}^2)}{t_{e(r)} - t_{s(r)}}
	\\
	g_r(t) &=& \frac{\frac{1}{2}(t^2 - t_{s(r)}^2) - t_{s(r)}(t - t_{s(r)})}{t_{e(r)} - t_{s(r)}}
\ea
The noiseless log-OD deterministic (given $x_0, \mu_1, \mu_2$). If  we further assume that the measurement noise is uncorrelated between different time points, then each $x_n$ becomes independent from every other $x_{n'}$. Assuming Gaussian noise with strength $\sigma_x$, we can write their generating distribution as
\be
	P(x^{(r)}\;|\;t^{(r)}, x_{r,0}, \mu_{r,1}, \mu_{r,2}) = \prod_{n=s(r)}^{e(r)} \text{Normal}\Big(x_n\;\Big|\;\text{mean}=x_r^\text{noiseless}(t_n), \;\text{variance}=\sigma_x^2\Big)
\ee









\end{document}